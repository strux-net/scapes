% Autogenerated from tex_tru using style ../test.txtTR
%
% do not edit this file, look for above mentioned and change this
% or you may loose all your changes and probably choose the wrong method
% Meta charset : UTF-8
% Meta author : Michael Naumann
% Meta m1 : cont for m1
% Meta m2 : cont for m2
\documentclass[10pt,a4paper]{article}
\usepackage[utf8x]{inputenc}
\parindent=0mm
\topskip=0mm
\pagestyle{headings}
\usepackage{german}
\usepackage[
    pdftitle={demo for tex_tru},
    pdfsubject={Eine kurze Beschreibung, worum es geht},
    pdfauthor={Michael Naumann},
    pdfkeywords={meta, information, pdf, hyperref, latex},
]{hyperref}
\usepackage{supertabular}
\usepackage{float}
\begin{document}
\begin{titlepage}
  \vfill
  \hfill\Huge{A test document}

  \bigskip
  \bigskip
  \hfill\Large{Version 0.00.012}
  \vfill
  \hfill{\sl Internal use only}
\end{titlepage}
\begin{tabular}{|l|p{3in}|}
  \hline
  Verfasser & Michael Naumann \\
  \hline
  Freigabe durch & Jemand anderes \\
  \hline
  Status & draft \\
  \hline
\end{tabular}

\bigskip

\bigskip

\begin{tabular}{|l|l|l|p{3in}|}
  \hline
  \multicolumn{4}{|c|}{\sc Version-Control} \\
  \hline
  {\bf Version} & {\bf Date} & {\bf Author} & {\bf Changes} \\
  &&&\\
  \hline
  &&&\\
  \hline
  &&&\\
  \hline
  &&&\\
  \hline
  &&&\\
  \hline
  &&&\\
  \hline
  &&&\\
  \hline
  &&&\\
  \hline
  &&&\\
  \hline
  &&&\\
  \hline
  &&&\\
  \hline
\end{tabular}
\newpage
\tableofcontents
\section{1st chapter}
\label{1st20chapter}%1st chapter
%
\subsection{1st section of 1st chapter}
\label{1st20section20of201st20chapter}%1st section of 1st chapter
%
1st line of 1st section of 1st chapter.

\begin{itemize}
\item{1st autolistline
}
\item{2nd autolistline
}
\end{itemize}
2nd line of 1st section of 1st chapter.

3rd line of 1st section of 1st chapter.

Links are possible, as in:

\begin{itemize}
\item{es folgt ein langer link \cite{nach hier}  und noch ein langer link \cite{nach dort} 
}
\item{a link to an external page http://www.namsu.de/Extra/befehle/Umlaute.html \cite{http://www.namsu.de/Extra/befehle/Umlaute.html}  here
}
\item{a link to \#3rd Chapter [\ref{3rd20Chapter}]  here
}
\item{a link to \#explicit mark 1 [\ref{explicit20mark201}]  here
}
\item{a link to \#an invalid local ref [\ref{an20invalid20local20ref}]  here
}
\end{itemize}
%
\subsection{2nd section of 1st chapter}
\label{2nd20section20of201st20chapter}%2nd section of 1st chapter
%
1st line of 2nd section of 1st chapter.

2nd line of 2nd section of 1st chapter.

3rd line of 2nd section of 1st chapter.

4th line of 2nd section of 1st chapter very very  very  very  very  very  very  very  very  very  very  very  very  very  very  very  very  very  very  very  very  very  very  very  very  very  very  very  very  very  very  very  very  very  very  very  very  very  very  very  very  very  very  very  very  very  very  very  very  very  very  very  very  very  very  very  very  very  very  very  very  very  very  very  very  very  very  very  very  very  very  very  very  very  very  very  very  very  very  very  very  very  very  very  very  very  very  very  very  very  very  very  very  very  very  very  very  very  very  very  very long.


5th line of 2nd section (after an empty line).

%
%
\section{2nd chapter}
\label{2nd20chapter}%2nd chapter
%
1st line of 2nd chapter.

2nd line of 2nd chapter.

3rd line of 2nd chapter.

Characters survive : \^{}!''¶\$\%\&/()=?`'@*+\~{}'\#$<$BR$>$$|$,;.:-\_äöü\{\}$[$$]$$\backslash$

\subsection{1st section of 2nd chapter}
\label{1st20section20of202nd20chapter}%1st section of 2nd chapter
%

\begin{itemize}
\item{1st autolistline
}
\begin{itemize}
\item{1st inner autolistline
}
\item{2nd inner autolistline
}
\end{itemize}
\item{2nd autolistline
}
\\
\item{3rd autolistline (with empty line before thisone)
}
\\
\end{itemize}
%
\subsection{2nd section of 2nd chapter (deals with explicit lists)}
\label{2nd20section20of202nd20chapter2028deals20with20explicit20lists29}%2nd section of 2nd chapter (deals with explicit lists)
%
\begin{itemize}
\item{1st explicit listline (empty line follows)
}
\\
\item{2nd explicit listline
}
\item{3rd explicit listline
}
\end{itemize}
%
\subsection{3rd section}
\label{3rd20section}%3rd section
%
\subsubsection{subsubsection}
\label{subsubsection}%subsubsection
%
\paragraph{paragraph}
\label{paragraph}%paragraph
%
\subparagraph{subparagraph}
\label{subparagraph}%subparagraph
%
\mbox{}\\\rule{0pt}{0pt}\\\begin{supertabular}{|l|l|l|l|l|l|l|}\hline
A table
& at a deep indent
\\\hline
with
& 2 lines
\\\hline
\end{supertabular}\\\\
\subparagraph{subparagraph1 too deep}
\label{subparagraph120too20deep}%subparagraph1 too deep
%
\subparagraph{subparagraph2 too deep}
\label{subparagraph220too20deep}%subparagraph2 too deep
%
tex cannot handle this deeply nests

%
%
%
%
%
%
\subsection{4th section}
\label{4th20section}%4th section
%
introducing a verbatim part

\begin{verbatim}

        can start with a blank line
must start at or after the next tab-position
  can continue with any blanks, eg. one
    or three
can contain	tabs
  1	one
  22	twenty two
  333	threehundredandthirtythree
but tabhandling is poor with tex_tru

can contain blank lines (one before and two to follow)


\end{verbatim}
%
%
\section{3rd Chapter}
\label{3rd20Chapter}%3rd Chapter
%
An autotable within a list

\begin{itemize}
\item{1st autolistline


is extended

}
\item{2nd autolistline
}
\mbox{}\\\rule{0pt}{0pt}\\\begin{supertabular}{|l|l|l|l|l|l|l|}\hline
1st
& autotableline
\\\hline
2nd
& autotableline
\\\hline
\end{supertabular}\\\\
\item{3rd autolistline
}
\end{itemize}
An autotable can jump to next page (since supertabular) with tex\_tru if in pre-mode and \$LD is $\backslash$$\backslash$$\backslash$$\backslash$. This is very annoying.

But since I changed $\backslash$$\backslash$$\backslash$$\backslash$ to $\backslash$n this phenomenon is not longer there. Smile.

\mbox{}\\\rule{0pt}{0pt}\\\begin{supertabular}{|l|l|l|l|l|l|l|}\hline
t
& 0
\\\hline
a1
& a2
\\\hline
b1
& b2
& b3
\\\hline
c1
&  
& c3
\\\hline
%
\end{supertabular}\\\\
explicit table

\mbox{}\\\rule{0pt}{0pt}\\\begin{supertabular}{|l|l|l|l|l|l|l|}\hline
h1
& h2
\\\hline
\hline
a1
& a2
\\\hline
b1
& b2
& b3
\\\hline
c1
\\\hline
%
\end{supertabular}\\\\
autoenum

\begin{enumerate}
\item{one
}
\item{two
}
\\
\item{three (with empty line between 2 and 3 and an empty line after)
}
\\
\item{another enum line


extended

}
\end{enumerate}
explicit enum

\begin{enumerate}
\item{one
}
\item{two
}
\\
\item{three (with empty line between 2 and 3 and an empty line after)
}
\\
\item{another enum line


extended

}
\end{enumerate}
some fancy code

\mbox{}\\\rule{0pt}{0pt}\\\begin{supertabular}{l|ll}
1 & {\tt stmt1	comment1}\\
2 & {\tt \ \ stmt2	comment2}\\
3 & {\tt \ \ an empty line follows}\\
4 & {\tt \ \ }\\
5 & {\tt stmt3}\\
6 & {\tt }\\
\end{supertabular}\rule{0pt}{0pt}\\\\
\label{explicit20mark201}
some definitions

\begin{description}
\item[def1:]{is1
}
\item[def2:]{is2
is3


}
\end{description}
{\sl this is a native
tex-statement}
Two Pictures

\begin{figure}[H]
\label{This20is20the20label20of20the20picture}
\pdfximage width \textwidth{../testpic.jpg}\pdfrefximage\pdflastximage
\caption{
This is the caption of the picture with a link to \#3rd Chapter [\ref{3rd20Chapter}]  here
}
\end{figure}
\begin{figure}[H]
\label{This20is20the20label20of20the20picture}
\pdfximage width \textwidth{../testpic_with_lnr.jpg}\pdfrefximage\pdflastximage
\caption{
The same picture, this time with linenumbers
}
\end{figure}
Another Picture

\begin{figure}[H]
\pdfximage width \textwidth{../anotherpic.jpg}\pdfrefximage\pdflastximage

\caption{
This picture cannot be generated, there are no : - lines
}
\end{figure}
\subsection{Now a final section}
\label{Now20a20final20section}%Now a final section
%
a section with a final statement

%
%
\begin{thebibliography}{strux}
  \bibitem{http://www.namsu.de/Extra/befehle/Umlaute.html} http://www.namsu.de/Extra/befehle/Umlaute.html
  \bibitem{nach dort} langer link
  \bibitem{nach hier} langer link
\end{thebibliography}
{\bf Here the document ends}
\listoftables
\listoffigures
\end{document}
%
